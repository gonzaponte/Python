
\documentclass[10pt]{article}
\usepackage{amsmath}
\usepackage{amsfonts}
\usepackage{amssymb}
\usepackage{graphicx}
\begin{document}

En el siguiente ejemplo vemos el ajuste de la atenuaci\'on gamma frente al espesor m\'asico del plomo atenuante. \\
\includegraphics[width=\textwidth]{grafica.pdf}

Las medidas fueron las siguientes \\
\begin{centering}
\begin{tabular}{|c|c|c|c|c|}
\hline
$x_{m} (g/cm^2)$ & N & t (s) & n & a \\ \hline $6.80$ & $1.61\cdot 10^{3}$ & $7.10\cdot 10^{1}$ & $\left(2.263\pm0.056\right)\cdot 10^{1}$ & $\left(5.43\pm0.19\right)\cdot 10^{-1}$\\\hline $1.08\cdot 10^{1}$ & $1.61\cdot 10^{3}$ & $1.31\cdot 10^{2}$ & $\left(1.228\pm0.031\right)\cdot 10^{1}$ & $\left(3.77\pm0.16\right)\cdot 10^{-1}$\\\hline $1.59\cdot 10^{1}$ & $9.05\cdot 10^{2}$ & $1.06\cdot 10^{2}$ & $8.54\pm0.28$ & $\left(2.73\pm0.11\right)\cdot 10^{-1}$\\\hline $1.93\cdot 10^{1}$ & $9.07\cdot 10^{2}$ & $1.47\cdot 10^{2}$ & $6.17\pm0.20$ & $\left(2.169\pm0.089\right)\cdot 10^{-1}$\\\hline $2.27\cdot 10^{1}$ & $9.28\cdot 10^{2}$ & $1.89\cdot 10^{2}$ & $4.91\pm0.16$ & $\left(1.803\pm0.075\right)\cdot 10^{-1}$\\
\hline
\end{tabular}
\end{centering}
\end{document}
